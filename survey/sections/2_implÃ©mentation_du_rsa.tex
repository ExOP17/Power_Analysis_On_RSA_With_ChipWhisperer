Après vous avoir introduit le contexte du projet, nous allons maintenant vous expliquer comment nous avons fait pour implémenter le crypto-système RSA.

\subsection{Implémentation sur 32 bits}
Comme nous l'avons vu dans la section précédente, pour implémenter le RSA, nous avons besoin des algorithmes suivant :
\begin{itemize}
	\item Exponentiation modulaire (déterminer $a$ tel que $a \equiv x^n \bmod n$);
	\item Inverse modulaire (étant donné $b$, trouver $a$ tel que $ab \equiv 1 \bmod n$).
\end{itemize}
Voici le pseudo code de ces deux algorithmes :

\begin{algorithm}[H]
\SetAlgoLined
\KwIn{base, exponent, modulus}
\KwOut{result}
$result \leftarrow 1$\;
$base \leftarrow base \mod modulus$\;
\While{$exponent > 0$}{
  \If{$exponent$ est impair}{
    $result \leftarrow (result \times base) \mod modulus$\;
  }
  $exponent \leftarrow exponent \div 2$\;
  $base \leftarrow (base \times base) \mod modulus$\;
}
\KwRet{$result$}\;
\caption{Exponentiation Modulaire}
\end{algorithm}

\begin{algorithm}[H]
\SetAlgoLined
\KwIn{a, m}
\KwOut{inverse modulaire}
$m0 \leftarrow m$\;
$y \leftarrow 0$\;
$x \leftarrow 1$\;
\If{$m = 1$}{
  \KwRet{$0$}\;
}
\While{$a > 1$}{
  $q \leftarrow a \div m$\;
  $t \leftarrow m$\;
  $m \leftarrow a \mod m$\;
  $a \leftarrow t$\;
  $t \leftarrow y$\;
  $y \leftarrow x - q \times y$\;
  $x \leftarrow t$\;
}
\If{$x < 0$}{
  $x \leftarrow x + m0$\;
}
\KwRet{$x$}\;
\caption{Inverse Modulaire}
\end{algorithm}
Notre implémentation en C de ces deux algorithmes se trouve dans l'archive final du projet. Une fois ces primitives implémentées, il a été facile d'implémenter la génération de clef, le chiffrement et le déchiffrement, vous pouvez également retrouver ces implémentations en C dans l'archive final du projet.

\subsection{Objectif 2048 bits}
\subsection{Implémentation du big int}
\subsection{Conclusion}
