\subsection{Définitions}
\begin{description}
    \item[BigInt] : Grand entier en français, généralement représenté sur plusieurs mots mémoire.
    \item[Chunk] : Une partie d'une entité, dans notre cas, un groupe de bits qui est utilisé pour stocker une partie d'un grand entier.
    \item[Débordement] : Un événement se produisant lorsqu'un calcul dépasse les capacités de stockage d'un type de données, nécessitant une gestion spéciale.
    \item[ANSSI] : Agence Nationale de la Sécurité des Systèmes d'Information
\end{description}

\subsection{Notations}
\begin{description}
    \item[$n$] : désigne le modulo utilisé pour les opérations modulaires
    \item[$p~\normalfont{et}~q$] : désigne deux nombres premiers tel que $n = p \times q$
    \item[$e$] : désigne l'exposant public
    \item[$d$] : désigne l'exposant privé
    \item[$m$] : désigne un message sous forme de nombre, tel que $m \in \{0, 1, \ldots, n-1\}$
    \item[$c$] : désigne le chiffré sous forme de nombre, tel que $c \in \{0, 1, \ldots, n-1\}$
    \item[$\phi(n)$] : désigne l'indicatrice d'Euler
\end{description}

