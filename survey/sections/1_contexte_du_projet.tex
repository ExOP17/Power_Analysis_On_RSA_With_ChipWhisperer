\subsection{Introduction}
En terme de sécurité informatique les méthodes de chiffrement sont très importantes notamment dans le domaine des télécommunication. La cryptographie est donc une notion très importante dans la cybersécurité actuelle. Les 3 principes de la cryptographie sont la confidentialité, l'authenticité et l'intégrité.
La confidentialité assure que le contenu d'un message chiffré ne peut être lu que par son destinataire.
L'authenticité assure l'origine du message, c'est à dire l'identité du messager.
Enfin l'intégrité assure la non-modification d'un message.
Dans notre cas nous mettrons en oeuvre des moyens pour s'attaquer à la confidentialité d'un protocole cryptographique. 
Ainsi nous allons vous présenter un algorithme connu pour sa robustesse et utilisé dans le monde entier, le protocole RSA.
\newpage
\subsection{Généralités sur le RSA}
Nous ne pouvons pas commencer la présentation du projet sans présenter le protocole RSA.
Le protocole RSA utilise une clé publique pour chiffrer le message et une clé privé pour le déchiffrer. Les clés publique et privé n'étant pas les mêmes, on parle de chiffrement asymétrique.
La robustesse du RSA repose dans la difficulté de factoriser un produit de deux grands nombres premiers.
Sans rentrer dans le détail de sa robustesse nous allons vous expliquer son fonctionnement:
\\

\textbf{Génération des clés}


Soit $p$ et $q$ deux nombres premiers distincts. On pose alors $n=pq$ et $\phi(n)=(p-1)(q-1)$.
On choisit ensuite un nombre $e$ premier avec $\phi(n)$ et strictement inférieur à $\phi(n)$
On peut enfin calculer $d$ l'inverse modulaire de $e$ modulo $\phi(n)$.
Ainsi le couple (n,e) constitue la clé publique et $d$ la clé privée.
\\

\textbf{Chiffrement}


Soit un message représenté par un entier naturel M strictement inférieur à n et C le message chiffré.
Nous avons la relation suivante:
\begin{equation}
\label{eq:chiffrement}

C \equiv M^e \pmod{n}

\end{equation}
Ainsi pour chiffrer un message M on utilise une fonction d'exponentiation modulaire utilisant la clé publique.
\\

\textbf{Déchiffrement}


Nous pouvons de la même manière déchiffrer le message chiffré à l'aide de la clé privée. 
Nous avons ainsi la relation suivante :
\begin{equation}
\label{eq:dechiffrement}

M \equiv C^d \pmod{n}

\end{equation}


Ainsi nous remarquons que la base du chiffrement et du déchiffrement repose sur une exponentiation modulaire. Le problème de l'exponentiation utilisé par le RSA est qu'elle ne peut pas être naïve, c'est à dire qu'on ne peut pas faire $ x^n = x*x*x...*x$ n fois. En effet, le RSA utilise des nombres de 2048 bits ce qui rend impossible l'utilisation d'un algorithme naïf.
Nous verrons par la suite lors de l'implémentation du RSA qu'on peut utiliser l'algorithme d'exponentiation rapide pour des nombres de 2048 bits.
\newpage
\subsection{Attaques par canaux auxiliaires}
Il est désormais temps de vous introduire au concept d'attaque par canaux auxiliaires.
Selon Wikipedia une attaque par canaux auxiliaires est une:
"Attaque informatique qui, sans remettre en cause la robustesse théorique des méthodes et procédures de sécurité, recherche et exploite des failles dans leur implémentation, logicielle ou matérielle."
En reprenant ce que nous vous avons présenter précédemment nous ne remettons pas en compte la robustesse du protocole RSA mais son implémentation utilisant l'algorithme d’exponentiation rapide.
Ainsi il existe une multitude d'attaques par canaux auxiliaires comme les attaques temporelles basées sur le temps mis par l'algorithme pour effectuer certaines opération, les attaques pas sondage qui consiste à analyser un circuit en y posant une sonde ou encore les attaques par consommation de courant.
Dans notre cas nous allons nous intéresser aux attaques par consommation de courant. Ces attaques sont basées sur le fait que chaque type d'opération que peut effectuer le processeur utilise plus ou moins de courant. On peut également généraliser ce principes pour les différentes fonctions d'un programme qui n'ont pas la même consommation électrique.

\subsection{Matériel utilisé}
Nous pouvons désormais présenter le matériel utilisé dans ce projet. Nous utiliserons un kit ChipWhisperer de la marque NewAE prêtée par notre tuteur Guy Goniat. Ce kit est composé de trois cartes différentes :
\begin{itemize}
\item Une carte cible qui fera tourner le firmware cible, c'est à dire le firmware implémentant le RSA
\item Une carte mère CW308 sur laquelle nous branchons 

\end{itemize}

\subsection{Enjeux et objectifs du projet}
Nous connaissons désormais tous les éléments nécessaires à la compréhension des enjeux et des objectifs de notre projet. L'objectif principal de ce projet est de récupérer la clé privée d'un utilisateur sans la connaitre, tout en restant dans un cadre légal. De ce fait nous attaquerons notre propre plateforme dédiée aux attaques par canaux auxiliaires. Les enjeux liés à ce projet sont nombreux, il faudra réussir à s’organiser autour d'un projet de groupe, comprendre

